\section{Conclusion \& Future Discussion}\label{sec:08}

  Currently, the platform already provides a huge improvement to the process of business application development.
  Instead of ``developing software'' we like saying ``modeling information system'' as every business represents an information system with information exchange occuring between different system actors such as managers, accoutants, engineers etc.
  The platfrom raised the level of abstraction from pure low-level programming terms such as ``data field'', ``table'', ``variable'' to a higher level understood by both business domain exprests and the developers.
  
  However, there is still a lot of work ahead of us.
  There are several complementing future directions ranging from specific developer-oriented enhancement to business decition support mechanisms.
  For example from the modeling perspective, right now there is no speficially targeted platfrom functionality to facilitate implementation of workflow oriented business processes\footnote{Having a general purpose programmig language at its hart, the platfrom does not inhibit the development of a custom workflow oriented process as part of a specific business application.}.
  There are already prototypes for such support, revolving around the Finite Automata Theory and Petri Nets, which is planned to be provided as part of the future platform versions.
  
  The future domain-oriented auditing system will provide an easy way not only to identify who changed what at the data level, but also a capability to analyse changes in terms of the domain semantics represneted by the object graph.  
  The next generations of the platform should incorporate fully domain-aware decision making mechanisms based on Decision Trees and Search Algorithms.

  For developers, the next platform generations should include the development tools in a form of IDE plugins to further simplify many aspects of TG-based application development as well as to greatly improve design time type-safety by tapping into the power of Java Pluggable Annotation Processing API.
  We're also looking forward to future Java enhancements such as Java Module System (aka Project Jigsaw) and Java FX to provide business application users with the best UX of Rich Internet Applications.