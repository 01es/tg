\section{Conclusion \& Future Discussion}\label{sec:08}

  This document strives to succinctly outline the most important principles and technological innovations of the Trident Genesis platform.
  The following items summarise the most essential platform characteristics: 
  \begin{itemize}
    \item The uniform development model and architecture raise the conceptual level of programming, providing a way for software developers to concentrate on developing business solutions instead of dealing with low-level technical details.
    \item A complete application development life-cycle starting with creation of data structures and finishing with UI construction and deployment follows a uniform system of concepts.
	This significantly speeds up the training of developers and improves their efficiency.
    \item The platform provides ready-to-use business domain modelling solutions essential for the development of any business application.
	  This includes seamless transition between object-oriented and relational data models, web communication, application deployment and support mechanisms, and more.
	  Developers do not need to learn different technologies and force them to properly work together to address these tasks.
    \item The exposure and uniform use of the business domain model at both the development and user interface levels establishes a ubiquitous language for communication between software developers, domain experts and users.
	  Such semantic transparency helps communication barriers between different groups of stakeholders.
  \end{itemize}

  The platform is dynamically evolving to enrich and further improve both the development experience and functionality of the business applications.
  There are several complementing future directions for the platform, ranging from specific developer-oriented enhancements to advanced business decision support mechanisms.
  For example from the modelling perspective, right now there is no specifically targeted platform functionality to facilitate implementation of work-flow oriented business processes\footnote{Having a general purpose programming language at its heart, the platform does not inhibit the development of custom work-flow oriented processes as part of specific business applications.}.
  There are already prototypes for such support, revolving around the Finite Automata Theory and Petri Nets, which are planned to be provided as part of future platform versions.
  
  The planned domain-oriented auditing system will provide not only an easy way to identify who changed what at the data level, but also the capability to analyse changes in terms of the domain semantics represented by the object graph.  
  The next generations of the platform should also incorporate domain-aware decision making mechanisms based on Decision Trees and Search Algorithms.

  For developers, the next platform generations should include development tools in a form of IDE plug-ins to further simplify many aspects of TG-based application development, as well as to greatly improve design time type-safety by tapping into the power of the Java Pluggable Annotation Processing API.
  We're also looking forward to future Java enhancements such as Java Module System (aka Project Jigsaw) and Java FX, to provide business application users with the best user experience of Rich Internet Applications.