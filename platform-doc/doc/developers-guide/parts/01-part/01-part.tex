\begin{partbacktext}
\part{Domain Driven Design with TG}  
  This part covers in-depth the provided by the platform development model, methodology, core components, information flow and API.
  We've tried to stay on point linking together the developed modelling mechanisms with practical examples.
  
  There two main approaches to describe software architecture -- top-down and bottom-up\footnote{\url{http://en.wikipedia.org/wiki/Top-down_and_bottom-up_design}}.
  In this and the following parts of the book we should utilise the bottom-up approach by discussing the parts of the puzzle, which will be put together in order to construct a final picture of the platform in a form a relatively complex application.

  A high-level platform overview is provided as a separate document, which can be downloaded \href{http://www.fielden.com.ua/trac/pnl-tg/attachment/wiki/WikiStart/architecture-overview.pdf}{separately}.

  All examples in this part build on the project generated and modified in part~\ref{part-I} of the book.
  Therefore, in case it was skipped it makes sense to go back, generated the training project and perform its modifications as described in chapter~\ref{ch00:03}.
\end{partbacktext}
