\chapter{Maven to Rule'em All or Project Comprehension with Maven}\label{ch00:01}

\myepigraph{One must learn by doing the thing; for though you think you know it, you have no certainty, until you try.}{Sophocles}                                                   

%\myepigraph{In our profession, precision and perfection are not a dispensable luxury, but a simple necessity.}{Niklaus Wirth}

\section{A Short Introduction to Apache Maven}
  
  Apache Maven is a software project management and comprehension tool. 
  It is based on the concept of a \emph{project object model} or simpy POM to understand the structure of the project. 
  Maven can manage a project's dependencies and build it from a central piece of information declared in its POM file.
  POM files are maintained by developers and changed as project demands.

  The major feature of Maven we would like to emphasise specifically is the dependency management.
  Every Java project relies on programming libraries that are provided as part of Java Runtime Environment (JRE) as well as the ones external to JRE.
  These external libraries, which could be inhouse or provided by 3rd parties, represent the dependencies projects depend upon.
  All Java libraries are packaged into jar files, which are effectively zip archives containing all library related resources such as compiled class files, image files and more.
  Thus, \emph{project dependency} is always a jar file.

  There are two main question about dependencies: 
  \begin{itemize}
    \item Where the dependency is located?
    \item What dependency version should be use?
  \end{itemize}

  In the old days all dependecies would be stored in a source version control systems together with the project's srouce.
  This way, any developer would always be able to get all project dependencies from one location.
  However, this approach has numerous issues.
  For example, several projects that depend on the same set or subset of libraries would keep duplicate copies in their respective version control repositories.

  As any project, programming libraries change over time and new versions may not neceserraly be backward compatible.
  This requires projects to know what dependency version should be used.
  So how should developers know what library version is used by their project and where to get, for example, the latest right version for a new project.

  These questions are amongst the main concerns addressed by Apache Maven.
  It is a very comprehensive tool that manges libraries, version resolution, localisation, and even handles transitive dependencies (i.e. libraries that depend on other libraries, which are not directly referenced by the project itself), which truly makes complex project possible.

  \subsection*{Repositories and Artifacts}
  All dependencies that are managed by Maven are identified by three parameters: group ID, artifact ID and version.
  Artifact ID and version information are encoded into the jar file names.
  For exaple, if the file name of some dependency is \texttt{platform-pojobl-\tikzinline{1.1}.jar} then the highlighted portion \texttt{\tikzinline{1.1}} indicates the version, and the preceeding part \texttt{platform-pojobl} is the artifact ID.
  The group ID corresponds to a directory structure of the repository where dependencies are stored.

  Maven repositories represent physical locations where dependencies are stored.
  Roughly speaking, there are two types of repositories -- local and remote.
  

  \paragraph*{Local Repository}
  The local repository is represented by a specially designated directory \texttt{$\sim$/.m2} on every development workstation where symbol \tikzinline{$\sim$} indicates a user home directory.
  Figure \ref{img:ch00:01:local_repository_strucutre} depicts a fragment of the local repository structure.
  The top level of the local repository (indicated as left most node \texttt{$\sim$/.m2}) consist of subdirectory \texttt{repository} and a configuraton file \texttt{.settings}, which is discussed in more details lated in this section.
  The \texttt{repository} directory consists of multiple subdirectories that represent group IDs of different dependencies.
  The next level corresponds to artifact IDs, which may contain multiple subdirectories indicating various versions of the same artifact.
  The lowest level of the directory tree corresponds to artifact files of a specific version.
  In this example, three files are presented:
  \begin{enumerate}
    \item \texttt{platform-pojo-bl-1.1.jar} -- the actual dependency with all the necessary resources such as compiled class files and images.
    \item \texttt{platform-pojo-bl-1.1-javadoc.jar} -- oprional Javadoc archive that contains API documentatino of the dependency, which is very useful during development process.
    \item \texttt{platform-pojo-bl-1.1.pom} -- POM of the dependency required to indentify all transitive dependecies.
  \end{enumerate}
  There could be some additional files associated with the dependency, which can be safely ignored at this stage.

  
  \begin{image}{A fragment of the local repository structure}{\label{img:ch00:01:local_repository_strucutre}}    
    \begin{tikzpicture}	[scale=0.9,
			  edge from parent fork right,grow=right,rounded corners=1pt,
			  every node/.style={scale=0.6,fill=white!10,rounded corners, level distance = 40mm},
			  edge from parent/.style={black,thick,draw},			  
			  level 1/.style={level distance=20mm,nodes={fill=white!10}, minimum width=20mm, minimum height=5mm},
			  level 2/.style={level distance=20mm,nodes={fill=white!10}, minimum width=20mm, minimum height=5mm},
			  level 3/.style={level distance=25mm,sibling distance=10mm, nodes={fill=white!10}, minimum width=30mm, minimum height=5mm},
			  level 4/.style={level distance=25mm,sibling distance=10mm, nodes={fill=white!10}, minimum width=20mm, minimum height=5mm},
			  level 5/.style={level distance=40mm,sibling distance=10mm, nodes={fill=white!10}, minimum width=50mm, minimum height=5mm}
			  ]      
      \node{$\sim$/.m2}
	child {node[fill=codebgcolor!30] {.settings}}
	child {node[fill=highlight] {repository}
	  child {node {group ID $N_1$}}	  
	  child {node {\ldots}}	  
	  child {node[fill=highlight] {fielden}
	    child {node {artifact ID $N_2$}}	    
	    child {node {\ldots}}	    
	    child {node[fill=highlight] {platfrom-pojobl}
	      child {node {version $N_3$}}
	      child {node {\ldots}}	      
	      child {node[fill=highlight] {1.1}
		child {node {\ldots}}
		child {node[fill=highlight] {platform-pojo-bl-1.1.pom}}
		child {node[fill=highlight] {platform-pojo-bl-1.1-javadoc.jar}}		
		child {node[fill=highlight] {platform-pojo-bl-1.1.jar}}
	      }
	      child {node {version 1}}
	    }
	    child {node {artifact ID 1}}	    
	  }	 
	  child {node {group ID 1}}	  
	};    
    \end{tikzpicture}
  \end{image}


  \paragraph*{Remote Repository}

  \subsection*{Download and Configure} 
  Maven can be downloaded from the official \href{http://maven.apache.org/download.html}{Apache Maven web site}\footnote{\url{http://maven.apache.org/download.html}}.
  The recommended at this time verion for TG-based applications is 2.2.1.
  Once downloaded please follow the \href{http://maven.apache.org/download.html#Installation}{installation instructions} also awailable on the official site\footnote{\url{http://maven.apache.org/download.html\#Installation}}.
  
  \begin{notebox}{JDK Version}{\label{mb:java}}
    It is important to note that Oracle JDK 6.x should be used for developing applications with TG, and it should be installed before downloading and installing Maven. 
  \end{notebox}
  
  One of the key 

  \lstset{language=XML,
	  escapechar=\%,
	  morekeywords={settings, encoding, mirrorOf, mirrors, mirror, id, profiles, profile, repositories, repository, url, releases, snapshots, version,
			pluginRepositories, pluginRepository, activeProfiles, activeProfile, servers, server, username, password, 
			directoryPermissions, filePermissions},
	  numbers=left, numberstyle=\tiny, basicstyle=\scriptsize\color{basiccolor}, stepnumber=1, numbersep=5pt, keywordstyle=\bfseries\color{codefgcolor}, stringstyle=\color{stringcolor}}
  \begin{code}{Maven Settings}{\label{lst:settings}}
    \begin{lstlisting}
      <?xml version="1.0" encoding="UTF-8"?>
      <settings>
	<mirrors>
	  <mirror>
	    <id>nexus</id>
	    <mirrorOf>*</mirrorOf>
	    <url>http://www.fielden.com.ua:8090/nexus/content/groups/public</url>
	  </mirror>
	</mirrors>
	<profiles>
	  <profile>
	    <id>nexus</id>
	    <repositories>
	      <repository>
		<id>central</id>
		<url>http://central</url>
		<releases><enabled>true</enabled></releases>
		<snapshots><enabled>true</enabled></snapshots>
	      </repository>
	    </repositories>
	    <pluginRepositories>
	      <pluginRepository>
		<id>central</id>
		<url>http://central</url>
		<releases><enabled>true</enabled></releases>
		<snapshots><enabled>true</enabled></snapshots>
	      </pluginRepository>
	    </pluginRepositories>
	  </profile>
	</profiles>

	<activeProfiles>
	  <activeProfile>nexus</activeProfile>
	</activeProfiles>

	<servers>
	  <server>
	    <id>nexus</id>
	    <username>your username</username>
	    <password>your password</password>
	    <directoryPermissions>0775</directoryPermissions>
	    <filePermissions>0664</filePermissions>
	  </server>
	  <server>
	    <id>Releases</id>
	    <username>your username</username>
	    <password>your password</password>
	    <directoryPermissions>0775</directoryPermissions>
	    <filePermissions>0664</filePermissions>
	  </server>
	  <server>
	    <id>Snapshots</id>
	    <username>your username</username>
	    <password>your password</password>
	    <directoryPermissions>0775</directoryPermissions>
	    <filePermissions>0664</filePermissions>
	  </server>
	</servers>
      </settings>
    \end{lstlisting}
  \end{code}


  \subsection*{POM Esseintials}  

  \hyperref[lst:pom]{Listing \ref{lst:pom}} illustrates an fragment of a POM file with relevant comments for each of the key tags relevant to the project.


  \clearpage
  \lstset{language=XML,
	  escapechar=\%,
	  morekeywords={project, modelVersion, groupId, artifactId, packaging, packaging, version, encoding, name, description},
	  numbers=left, numberstyle=\tiny, basicstyle=\scriptsize\color{basiccolor}, stepnumber=1, numbersep=5pt, keywordstyle=\bfseries\color{codefgcolor}, stringstyle=\color{stringcolor}}
  \begin{code}{Project Object Model (POM)}{\label{lst:pom}}
    \begin{lstlisting}
    <?xml version="1.0" encoding="UTF-8"?>
    <project xmlns="..."
	  xsi:schemaLocation="...">
	  <modelVersion>4.0.0</modelVersion>
	  %\tikzref{lst_pom_n1}{-1ex}{2ex}%<groupId>fielden</groupId>%\tikzref{lst_pom_n2}{1ex}{-0.8ex}\tikzref{lst_pom_n2_1}{1ex}{0.5ex}%
	  %\tikzref{lst_pom_n3}{-1ex}{2ex}%<artifactId>platform-parent</artifactId>%\tikzref{lst_pom_n4}{1ex}{-0.8ex}\tikzref{lst_pom_n4_1}{1ex}{0.5ex}%
	  <packaging>pom</packaging>
	  <version>1.1-SNAPSHOT</version>
	  <name>Trident Genesis Platform Parent</name>
	  <description>
		  Trident Genesis is an application platform designed for
		  development of ERP type applications supporting Client/Server
		  and
		  RIA/ROA architectures.
		  Its main goal is to provide simple integrated solution for 
		  development of information system with domain oriented approach. 
	  </description>
    </project> 
    \end{lstlisting}
  \end{code}

  \tikzhighlight{lst_pom_n1}{lst_pom_n2}  
  \tikznote{lst_pom_a1}{lst_pom_n2_1}{6cm}{0.5cm}{5cm}{Project Group}{This tag indicates project's group, which usually referes to the company or development group responsible for implementing the project.}

  \tikzhighlight{lst_pom_n3}{lst_pom_n4}
  \tikznote{lst_pom_a2}{lst_pom_n4_1}{4cm}{-0.5cm}{5cm}{Project Artifact}{There can be several notes for the same highlighted area.}